\documentclass{article}

\usepackage[T1]{fontenc}
\usepackage[utf8]{inputenc}

\usepackage{microtype,array,hyperref,xcolor,pgffor,enumitem}
\usepackage[main=ngerman,english]{babel}

\definecolor{cgreen}{RGB}{85,170,0}

\usepackage[a4paper,left=1.5cm,right=1.5cm,top=2cm,bottom=2cm]{geometry}

\usepackage{random-intropoem}

\hypersetup{%
    pdfborder=0 0 0,%
    colorlinks=true,%
    allcolors=cgreen!60!purple%
}

\usepackage{imakeidx}
\makeindex[title=Befehlsübersicht,columns=2,columnsep=0.75cm]\renewcommand{\indexname}{Befehlsübersicht}

\def\say#1{\glqq{#1}\grqq{}}

\def\cmd#1{\texttt{\textcolor{cgreen}{\textbackslash#1}}}
\def\cmdlink#1{\phantomsection\label{cmd:#1}\cmd{#1}}
\def\cmdref#1{\hyperref[cmd:#1]{\cmd{#1}}}

\def\arg#1{\textit{\textcolor{cgreen!90!white}{#1}}}
\def\manArg#1{\texttt{\{\,\arg{#1}\,\}}}
\def\optArg#1{\texttt{[\,\arg{#1}\,]}}

\def\secref#1{\hyperref[#1]{\ref*{#1} \nameref*{#1}}}

\long\def\imp#1{\emph{\textcolor{orange}{#1}}}

\newenvironment{command}[3][v1.0]{\leavevmode\\[-\baselineskip]\hspace*{-1.75em}\index{#2@\cmdref{#2}}\cmdlink{#2}$\,$#3\hfill{}\texttt{#1}\nopagebreak\smallskip\\*}{\bigskip\par{}}

\setlength{\parindent}{0pt}
\setlength{\parskip}{0.35\baselineskip plus 0.15\baselineskip minus 0.05\baselineskip}

\title{\textsf{\textcolor{orange}{\{}\textcolor{cgreen}{random-intropoem}\textcolor{orange}{\}}}}
\author{Florian Sihler}
\date{\today}

\begin{document}

\maketitle

\begin{abstract}
   Dieses Paket übernimmt die Random-Flavourtext-Teil von Lilly.
\end{abstract}

\tableofcontents

\section{Nutzermakros}

\begin{command}{GetRip}{\manArg{num}}
    Liefert den Text mit Nummer \arg{num}. Muss zwischen 1 und \GetMaxRip{} liegen. So liefert zum Beispiel der erste Text (zur Trennung wurde dieser mittels \cmd{fbox} umrahmt. Für die Breite siehe \cmdref{RipWidth}):\\
    \fbox{\GetRip{1}}

    Siehe \cmdref{GetMaxRip}, \cmdref{GetRandomRip} und \cmdref{SetRipStyle} für weitere Informationen.
\end{command}

\begin{command}{GetRipText}{\manArg{num}}
    Liefert den Text für die Zahl zwischen 1 und \GetMaxRip{}.
\end{command}

\begin{command}{GetRipAuthor}{\manArg{num}}
    Liefert den Autor für die Zahl zwischen 1 und \GetMaxRip{}.
\end{command}

\begin{command}{GetRipDate}{\manArg{num}}
    Liefert das Datum für die Zahl zwischen 1 und \GetMaxRip{}.
\end{command}

\begin{command}{GetRipName}{\manArg{num}}
    Liefert den Namen für die Zahl zwischen 1 und \GetMaxRip{}.
\end{command}

\begin{command}{GetMaxRip}{}
    Liefert die maximale Anzahl an registrierten Texten (\GetMaxRip).
\end{command}

\begin{command}{GetRandomRip}{}
    Liefert einen zufälligen Text mittels \cmdref{GetRip}. Beispiel (der Kasten wurde wieder mittels \cmd{fbox} gesetzt):\\
    \fbox{\GetRandomRip}.
\end{command}

\begin{command}{SetRipStyle}{\manArg{stil}}
Legt die Darstellung von \cmdref{GetRip} fest. Innerhalb können die Befehle \begin{itemize}[nolistsep]
    \item \cmd{riptext} für den Text (\cmdref{GetRipText}),
    \item \cmd{ripauthor} für den Autor (\cmdref{GetRipAuthor}),
    \item \cmd{ripdate} für das Erstellungsdatum (\cmdref{GetRipDate}) und
    \item \cmd{ripname} für den Namen des Textes (\cmdref{GetRipName})
\end{itemize}
verwendet werden. Die Standarddefinition ist die Folgende:\\
\makeatletter%
\def\clamp#1->#2\@nil{#2}
\parbox{0.85\linewidth}{\ttfamily\expandafter\clamp\meaning\rip@ripstyle\@nil}%
\makeatother%
\end{command}

\begin{command}{RipWidth}{\manArg{len}}
    Setzt die Breite der gesetzten Texte.
\end{command}

\section{Versionierung}
\subsection{Version 1.0}
Grundlegende Kommandos hinzugefügt. Darunter: \cmdref{GetRipText} und \cmdref{GetRandomRipText}.

\section{Alle Texte}

\foreach\i in {1, ..., \GetMaxRip} {%
\paragraph{Text \i}~\medskip\\
\GetRip{\i}%
}

\clearpage\appendix
\phantomsection\addcontentsline{toc}{section}{Befehlsübersicht}
\printindex

\end{document}